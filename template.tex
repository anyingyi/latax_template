\documentclass[12pt]{report}

\usepackage{ctex}
\usepackage[utf8]{inputenc}
\usepackage{graphicx}
\graphicspath{{images/}}
\usepackage[a4paper,width=180mm,top=25mm,
bottom=25mm]{geometry}
\usepackage{caption}
\usepackage{subcaption} %used for subfigure
\usepackage{fancyhdr}
\pagestyle{fancy}
\fancyhead{}
\fancyhead[]{thesis title}

\usepackage{listings}
\usepackage{xcolor}
\lstset{
    numbers=left, 
    numberstyle= \tiny, 
    keywordstyle= \color{ blue!70},
    commentstyle= \color{red!50!green!50!blue!50}, 
    frame=shadowbox, % 阴影效果
    rulesepcolor= \color{ red!20!green!20!blue!20} ,
    escapeinside=``, % 英文分号中可写入中文
    xleftmargin=2em,xrightmargin=2em, aboveskip=1em,
    framexleftmargin=2em,
	tabsize=4,
	language=c++
} 

\title{
    {Thesis Title}\\
    {\large Institution Name}\\
    {\includegraphics{piesat.jpg}}
}
\author{Author Name}
\date{Day Month Year}

\begin{document}

\maketitle

\chapter*{Abstract}
this is a template of learning latex.

\chapter*{Dedication}
To mum and dad

\tableofcontents
\listoffigures
\listoftables

\chapter{Introduction}
\section{space}
attention machnical is \  important.\par
attention machnical is \quad important.\par %use \quad to generate space.
attention machnical is \hfill  important.\par   % \hfill:adaptive space. 

\section{special mark}
\# \$ \% \{ \} \_{} \^{} \textbackslash \&

\section{引号}
`hello' ``hello'' ``  '' "hello"

\section{math formular}
exchange formular is $a+b=b+a$ \par
exchange formular is \begin{math}
    a+b=b+a
\end{math} \par
exchange formular is \(a+b=b+a\) \par
exchange formular is 
\begin{equation}
    a+b=b+a \label{eq:1}
\end{equation}

$$\alpha^{20}+\beta_{20}+\gamma^2=0$$

\section{希腊字母}
$\alpha$ $\beta$ $\gamma$ $\epsilon$ $\pi$
$\omega$ $\Gamma$ $\Delta$ 

\section{table}
this is the basic table of latex.
\begin{table}
    \centering
    \begin{tabular}{l | l | r}
        \hline
        a & b & c\\
        \hline
        1 & 2 & 3\\
        1 & 5 & 6\\
        \hline 
    \end{tabular}
    \caption{basic table}
    \label{basic_tab}
\end{table}

\section{subtable}
here is the subtable.
\begin{table}
    \begin{subtable}[h]{0.45\textwidth}
        \centering
        \begin{tabular}{ccc}\\
            xiao & wen & miao\\
            \hline
            98 & 95 & 100\\
            98 & 95 & 100\\
            98 & 95 & 100\\
            98 & 95 & 100\\
            \hline
            
        \end{tabular}
        \caption{first}
        \label{first}
    \end{subtable}
    \hfill
    \begin{subtable}[h]{0.45\textwidth}
        \centering
        \begin{tabular}{ccc}\\
            xiao & wen & miao\\
            \hline
            98 & 95 & 100\\
            98 & 95 & 100\\
            98 & 95 & 100\\
            98 & 95 & 100\\
            \hline
            
        \end{tabular}
        \caption{second}
        \label{second}
    \end{subtable}
    \caption{subfigure}
    \label{sub}
\end{table}

\section{figure}
安装好ACIS软件后,需要添加两个系统变量A3DT和ARCH。
A3DT的值设置为安装路径。
\begin{figure}
    \centering
    \begin{subfigure}[b]{0.3\textwidth}
        \centering
        \includegraphics[width=\textwidth]{piesat.jpg}
        \caption{piesat1}
        \label{piesat1}
    \end{subfigure}
    \begin{subfigure}[b]{0.3\textwidth}
        \centering
        \includegraphics[width=\textwidth]{piesat.jpg}
        \caption{piesat1}
        \label{piesat2}
    \end{subfigure}
    \begin{subfigure}[b]{0.3\textwidth}
        \centering
        \includegraphics[width=\textwidth]{piesat.jpg}
        \caption{piesat1}
        \label{piesat3}
    \end{subfigure}
    \caption{piesat}
    \label{piesat}
\end{figure}

\section{简单的ACIS程序}
\lstset{language=c++}
\begin{lstlisting}
#include <stdio.h>
#include "acis.hxx"
#include "kernapi.hxx"
#include"api.hxx"
#include"cstrapi.hxx"
#include"lists.hxx"
#include"alltop.hxx"
#include"get_top.hxx"
#include"spatial_license.h"
#include"license.hxx"
#include "spa_unlock_result.hxx"

using namespace std;

void do_something_cuboid();
void do_something();
int my_initialization();
int my_termination();



// The main program...
int main(int argc, char** argv) {

	int ret_val = my_initialization();
	if (ret_val)
		return 1;

	do_something_cuboid();

	ret_val = my_termination();
	if (ret_val)
		return 1;

	return 0;
}

void do_something()
{
	//your application code
	printf("hello world!\n");
}


int my_initialization()
{
	outcome result = api_start_modeller(0);
	if (!result.ok()) {
		err_mess_type err_no = result.error_number();
		printf("error in api_start_modeller() %d:%s\n", 
			err_no, find_err_mess(err_no));
		return err_no;
	}

	// This can be done right after calling api_start_modeller().
	spa_unlock_result out = spa_unlock_products(SPATIAL_LICENSE);

	api_initialize_constructors();

	return 0;
}

int my_termination()
{
	api_terminate_constructors();

	outcome result = api_stop_modeller();
	if (!result.ok()) {
		err_mess_type err_no = result.error_number();
		printf("error in api_stop_modeller() %d:%s\n",
			err_no, find_err_mess(err_no));
		return err_no;
	}

	return 0;
}
\end{lstlisting}

CPU Core i5-10500
显卡 AMD Radeon 520 ( 2 GB / 宝龙达 )
内存 16G
固态硬盘 240G
机械硬盘 1T

\chapter{chapter two}
\section{section title}
As we’ll see later in this example, I’ve already obtained the 
predicted bounding boxes from our five respective images and 
hardcoded them into this script to keep the example short and 
concise.


\section{section title}
As we’ll see later in this example, I’ve already obtained the 
predicted bounding boxes from our five respective images and 
hardcoded them into this script to keep the example short and
 concise.

\chapter{Conclusion}
we discuss CBAM(convolution block attention model). about 
channel attention model and spatial model. see \ref{fig:1}.
\begin{figure}[h]
    \centering
    \includegraphics{piesat.jpg}
    \caption{the picture of piesat}
    \label{fig:1}
\end{figure}[h]

\appendix
\chapter{appendix title}
we discuss CBAM(convolution block attention model). about 
channel attention model and spatial model.



\end{document}